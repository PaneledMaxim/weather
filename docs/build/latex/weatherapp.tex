%% Generated by Sphinx.
\def\sphinxdocclass{report}
\documentclass[letterpaper,10pt,russian]{sphinxmanual}
\ifdefined\pdfpxdimen
   \let\sphinxpxdimen\pdfpxdimen\else\newdimen\sphinxpxdimen
\fi \sphinxpxdimen=.75bp\relax
\ifdefined\pdfimageresolution
    \pdfimageresolution= \numexpr \dimexpr1in\relax/\sphinxpxdimen\relax
\fi
%% let collapsible pdf bookmarks panel have high depth per default
\PassOptionsToPackage{bookmarksdepth=5}{hyperref}

\PassOptionsToPackage{booktabs}{sphinx}
\PassOptionsToPackage{colorrows}{sphinx}

\PassOptionsToPackage{warn}{textcomp}
\usepackage[utf8]{inputenc}
\ifdefined\DeclareUnicodeCharacter
% support both utf8 and utf8x syntaxes
  \ifdefined\DeclareUnicodeCharacterAsOptional
    \def\sphinxDUC#1{\DeclareUnicodeCharacter{"#1}}
  \else
    \let\sphinxDUC\DeclareUnicodeCharacter
  \fi
  \sphinxDUC{00A0}{\nobreakspace}
  \sphinxDUC{2500}{\sphinxunichar{2500}}
  \sphinxDUC{2502}{\sphinxunichar{2502}}
  \sphinxDUC{2514}{\sphinxunichar{2514}}
  \sphinxDUC{251C}{\sphinxunichar{251C}}
  \sphinxDUC{2572}{\textbackslash}
\fi
\usepackage{cmap}
\usepackage[T1]{fontenc}
\usepackage{amsmath,amssymb,amstext}
\usepackage{babel}





\usepackage[Sonny]{fncychap}
\ChNameVar{\Large\normalfont\sffamily}
\ChTitleVar{\Large\normalfont\sffamily}
\usepackage{sphinx}

\fvset{fontsize=auto}
\usepackage{geometry}


% Include hyperref last.
\usepackage{hyperref}
% Fix anchor placement for figures with captions.
\usepackage{hypcap}% it must be loaded after hyperref.
% Set up styles of URL: it should be placed after hyperref.
\urlstyle{same}

\addto\captionsrussian{\renewcommand{\contentsname}{Содержание:}}

\usepackage{sphinxmessages}
\setcounter{tocdepth}{1}



\title{Weather App}
\date{нояб. 15, 2025}
\release{1.0.0}
\author{Maxim}
\newcommand{\sphinxlogo}{\vbox{}}
\renewcommand{\releasename}{Выпуск}
\makeindex
\begin{document}

\ifdefined\shorthandoff
  \ifnum\catcode`\=\string=\active\shorthandoff{=}\fi
  \ifnum\catcode`\"=\active\shorthandoff{"}\fi
\fi

\pagestyle{empty}
\sphinxmaketitle
\pagestyle{plain}
\sphinxtableofcontents
\pagestyle{normal}
\phantomsection\label{\detokenize{index::doc}}


\sphinxAtStartPar
Простое приложение для получения погоды с кэшированием.

\sphinxstepscope


\chapter{app package}
\label{\detokenize{app:app-package}}\label{\detokenize{app::doc}}

\section{Submodules}
\label{\detokenize{app:submodules}}

\section{app.api module}
\label{\detokenize{app:app-api-module}}

\section{app.cache module}
\label{\detokenize{app:module-app.cache}}\label{\detokenize{app:app-cache-module}}\index{module@\spxentry{module}!app.cache@\spxentry{app.cache}}\index{app.cache@\spxentry{app.cache}!module@\spxentry{module}}
\sphinxAtStartPar
Файловый кеш для хранения результатов запроса погоды.
\index{get() (в модуле app.cache)@\spxentry{get()}\spxextra{в модуле app.cache}}

\begin{fulllineitems}
\phantomsection\label{\detokenize{app:app.cache.get}}
\pysigstartsignatures
\pysiglinewithargsret{\sphinxcode{\sphinxupquote{app.cache.}}\sphinxbfcode{\sphinxupquote{get}}}{\sphinxparam{\DUrole{n}{key}\DUrole{p}{:}\DUrole{w}{ }\DUrole{n}{str}}}{{ $\rightarrow$ dict\DUrole{w}{ }\DUrole{p}{|}\DUrole{w}{ }None}}
\pysigstopsignatures
\sphinxAtStartPar
Получает значение из кеша, учитывая срок годности.
\begin{quote}\begin{description}
\sphinxlineitem{Параметры}
\sphinxAtStartPar
\sphinxstyleliteralstrong{\sphinxupquote{key}} (\sphinxstyleliteralemphasis{\sphinxupquote{str}}) \textendash{} Ключ записи.

\sphinxlineitem{Результат}
\sphinxAtStartPar
Данные, если запись актуальна, иначе None.

\sphinxlineitem{Тип результата}
\sphinxAtStartPar
Optional{[}dict{]}

\end{description}\end{quote}

\end{fulllineitems}

\index{set() (в модуле app.cache)@\spxentry{set()}\spxextra{в модуле app.cache}}

\begin{fulllineitems}
\phantomsection\label{\detokenize{app:app.cache.set}}
\pysigstartsignatures
\pysiglinewithargsret{\sphinxcode{\sphinxupquote{app.cache.}}\sphinxbfcode{\sphinxupquote{set}}}{\sphinxparam{\DUrole{n}{key}\DUrole{p}{:}\DUrole{w}{ }\DUrole{n}{str}}\sphinxparamcomma \sphinxparam{\DUrole{n}{data}\DUrole{p}{:}\DUrole{w}{ }\DUrole{n}{dict}}}{{ $\rightarrow$ None}}
\pysigstopsignatures
\sphinxAtStartPar
Сохраняет данные в кеш.
\begin{quote}\begin{description}
\sphinxlineitem{Параметры}\begin{itemize}
\item {} 
\sphinxAtStartPar
\sphinxstyleliteralstrong{\sphinxupquote{key}} (\sphinxstyleliteralemphasis{\sphinxupquote{str}}) \textendash{} Ключ записи.

\item {} 
\sphinxAtStartPar
\sphinxstyleliteralstrong{\sphinxupquote{data}} (\sphinxstyleliteralemphasis{\sphinxupquote{dict}}) \textendash{} Данные для сохранения.

\end{itemize}

\end{description}\end{quote}

\end{fulllineitems}



\section{app.commands module}
\label{\detokenize{app:app-commands-module}}

\section{app.parser module}
\label{\detokenize{app:module-app.parser}}\label{\detokenize{app:app-parser-module}}\index{module@\spxentry{module}!app.parser@\spxentry{app.parser}}\index{app.parser@\spxentry{app.parser}!module@\spxentry{module}}
\sphinxAtStartPar
Парсер командной строки для приложения погоды.
\index{build\_parser() (в модуле app.parser)@\spxentry{build\_parser()}\spxextra{в модуле app.parser}}

\begin{fulllineitems}
\phantomsection\label{\detokenize{app:app.parser.build_parser}}
\pysigstartsignatures
\pysiglinewithargsret{\sphinxcode{\sphinxupquote{app.parser.}}\sphinxbfcode{\sphinxupquote{build\_parser}}}{}{{ $\rightarrow$ ArgumentParser}}
\pysigstopsignatures
\sphinxAtStartPar
Создаёт и настраивает парсер аргументов CLI.
\begin{quote}\begin{description}
\sphinxlineitem{Результат}
\sphinxAtStartPar
Готовый объект парсера.

\sphinxlineitem{Тип результата}
\sphinxAtStartPar
argparse.ArgumentParser

\end{description}\end{quote}

\end{fulllineitems}



\section{Module contents}
\label{\detokenize{app:module-app}}\label{\detokenize{app:module-contents}}\index{module@\spxentry{module}!app@\spxentry{app}}\index{app@\spxentry{app}!module@\spxentry{module}}
\sphinxAtStartPar
Пакет weather — логика получения погоды, работа с кешем,
API open\sphinxhyphen{}meteo и поддержка CLI\sphinxhyphen{}команд.

\sphinxstepscope


\chapter{app}
\label{\detokenize{modules:app}}\label{\detokenize{modules::doc}}

\chapter{Индексы и поиск}
\label{\detokenize{index:id1}}\begin{itemize}
\item {} 
\sphinxAtStartPar
\DUrole{xref,std,std-ref}{genindex}

\item {} 
\sphinxAtStartPar
\DUrole{xref,std,std-ref}{modindex}

\item {} 
\sphinxAtStartPar
\DUrole{xref,std,std-ref}{search}

\end{itemize}


\renewcommand{\indexname}{Содержание модулей Python}
\begin{sphinxtheindex}
\let\bigletter\sphinxstyleindexlettergroup
\bigletter{a}
\item\relax\sphinxstyleindexentry{app}\sphinxstyleindexpageref{app:\detokenize{module-app}}
\item\relax\sphinxstyleindexentry{app.cache}\sphinxstyleindexpageref{app:\detokenize{module-app.cache}}
\item\relax\sphinxstyleindexentry{app.parser}\sphinxstyleindexpageref{app:\detokenize{module-app.parser}}
\end{sphinxtheindex}

\renewcommand{\indexname}{Алфавитный указатель}
\printindex
\end{document}